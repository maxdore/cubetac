\documentclass[11pt]{article}

\usepackage{hyperref}
\def\exampleautorefname{Example}

\usepackage{amsmath}
\usepackage{amssymb}
\usepackage{amsthm}
\theoremstyle{definition}
\newtheorem{definition}{Definition}
\newtheorem{theorem}{Theorem}
\newtheorem{example}{Example}
\newtheorem*{expcont}{Example \continuation}
\newcommand{\continuation}{??}
\newenvironment{examplecontd}[1]
{\renewcommand{\continuation}{\ref{#1}}\expcont[continued]}
{\endexpcont}
\usepackage{algorithm}
\usepackage[noEnd=True, indLines=True]{algpseudocodex}
\algrenewcommand\algorithmicrequire{\textbf{Input:}}
\algrenewcommand\algorithmicensure{\textbf{Output:}}
\def\algorithmautorefname{Algorithm}

\usepackage{ifthen}
\usepackage{tcolorbox}
\newcommand{\todo}[1]{
  \begin{tcolorbox}
    TODO {#1} 
  \end{tcolorbox}
}

\newcommand{\mdef}{:=}
\newcommand{\mname}[1]{\textit{{#1}}}
\newcommand{\problem}[1]{\textsc{{#1}}}
\newcommand{\mlist}[1]{[ {#1} ]}

\newcommand{\join}{\wedge}
\newcommand{\meet}{\vee}
\newcommand{\dedekind}{\square_{\join \meet}}
% \newcommand{\pint}{[1]}
\newcommand{\pint}[1]{\mathbf{1}^{#1}}
\newcommand{\pintrestr}[3]{\mathbf{1}^{#1}_{{#2}={#3}}}
\newcommand{\izero}{\mathsf{0}}
\newcommand{\ione}{\mathsf{1}}
\newcommand{\ivar}{*}
\newcommand{\restrict}[2]{{#1}|_{#2}}
\newcommand{\image}[1]{\textsf{Img}({#1})}
\newcommand{\psh}[1]{\mathbf{Set}^{{#1}^{op}}}
\renewcommand{\hom}[2]{\text{Hom}({#1} , {#2})}
\renewcommand{\dim}[1]{\mathsf{dim}({#1})}
\newcommand{\ctxtdim}[1]{|{#1}|}
\newcommand{\smap}[1]{s_{{#1}}}
\newcommand{\dmap}[2]{d_{({#1} , {#2})}}
% \newcommand{\cont}[2]{{#1}\langle{#2}\rangle}
\newcommand{\cont}[2]{ \ifthenelse{\equal{#2}{}}{#1}{{#1}\langle{#2}\rangle} }



\newcommand{\pow}[1]{\mathcal{P}({#1})}
\newcommand{\cset}[1]{\mathsf{{#1}}}
\newcommand{\boundary}[1]{\partial({#1})}
\newcommand{\comp}[2]{\mathsf{Comp}({#1}\ {#2})}

\newcommand{\substtwo}[2]{\tiny
  \arraycolsep=.4pt\def\arraystretch{1}
  \begin{array}{ll}
    0 &\mapsto {#1} \\
    1 &\mapsto {#2}
  \end{array}
}
\newcommand{\oneconst}{\substtwo{()}{()}}
\newcommand{\oneid}{\substtwo{0}{1}}

\newcommand{\substfour}[4]{\tiny
  \arraycolsep=.4pt\def\arraystretch{1}
  \begin{array}{ll}
    00 &\mapsto {#1} \\
    01 &\mapsto {#2} \\
    10 &\mapsto {#3} \\
    11 &\mapsto {#4} 
  \end{array}
}


\title{Searching in cubes}
\author{Maximilian Dor\'e}
\date{\today}


\begin{document}
\maketitle	

As observed by \cite{orton17_axiom_model_cubic_type_theor_topos}, do not require
a full De Morgan structure. In particular, it suffices to consider only $\meet$
and $\join$ forming a bounded free distributive lattice.

\cite{williamson12_combin_homot_theor}: we shall think geometrically, and prove
algebraically!

% Allowable boundary shapes are generating cofibrations from a model categorical perspective


\section{Cubical sets and their boundaries}

\subsection{Background}

The Dedekind cube category $\dedekind$ is the full subcategory of the category
of posets and monotone maps with objects $\pint{n}$ for $n \geq 0$, where $\pint{}
= \{ \izero < \ione \}$. Therefore morphisms in $\dedekind$ are of
the form $\pint{m} \to \pint{n}$. The cardinality of $\hom{\pint{m}}{\pint{}}$ is the
$m$-th Dedekind number, which explains the name of $\dedekind$.

We denote elements of $\pint{n}$ with $(e_1 \ldots e_n)$ where $e_i \in
\{\izero, \ione\}$ for $1 \leq i \leq n$ (we sometimes may omit the brackets).
For an element $x = (e_1 \ldots e_n)$ write $x_i \mdef e_i$. The poset
$\pint{0}$ has one element $()$. We will write $\pintrestr{n}{i}{e} \mdef
\{ x \in \pint{n} \mid x_i = e \}$, which is a subposet of $\pint{n}$. Given
$\pint{m} \to \pint{n}$ and a poset $P \subseteq \pint{m}$, we will denote with
$\restrict{\sigma}{P} : P \to \pint{n}$ the restriction of $\sigma$ to $P$.

Certain poset maps of interest are:
\begin{align*}
  s^i &: \pint{n} \to \pint{n-1}, (e_1 \ldots e_n) \mapsto (e_1 \ldots e_{i-1} e_{i+1} \ldots e_n) \text{ for } 1 \leq i \leq n\\
  d^{(i,e)} &: \pint{n-1} \to \pint{n}, 
      (e_1 \ldots e_{n-1}) \mapsto (e_1 \ldots e_{i-1} e e_{i+1} \ldots e_{n-1}) \text{ for } 1 \leq i \leq n, e \in \{\izero,\ione\}
\end{align*}


Cubical sets are objects of $\psh{\dedekind}$, i.e., presheaves over the
Dedekind cube category. For a cubical set $X$ write $X_n \mdef X(\pint{n})$,
which are called the $n$-cells of $X$. Given an element $p$ of $X_n$, we call
$\dim{p} \mdef n$ the dimension of $p$. If the cubical set $X$ is understood,
write $\smap{i} \mdef X(s^i) : X_{n-1} \to X_n$ and $\dmap{i}{e} \mdef
X(d^{(i,e)}) : X_n \to X_{n-1}$, these maps are called \mname{degeneracy maps}
and \mname{face maps}, respectively.

The objects in the image of some $\smap{i}$ are called \mname{degeneracies}. If
the dimension $n$ is understood, we will sometimes denote with $\smap{m}$ the
composite $\smap{m} \circ \smap{m-1} \circ \ldots \circ \smap{n+1}$. This gives
the shorthand $\smap{m}(p)$ for an $n$-cell $p$ considered a degenerate
$m$-cell.

Conversely, the cell $\dmap{i}{e}(p)$ is called the $(i,e)$-th face of $p$. We
will call the collection of all faces of an $n$-cell $p$ its boundary $\boundary{p}$:
$$\boundary{p} \mdef \mlist{ \dmap{n}{0}(p), \dmap{n}{1}(p) , \ldots , \dmap{1}{0}(p), \dmap{1}{1}(p)}$$

The functor laws of $X$ ensure that the faces of $p$ have overlapping
boundaries. TODO EXPLAIN/EXAMPLE

% Note that in any non-empty cubical set $X$, we must have non-empty $X_n$ for
% all $n$: given an $n$-cell $p$, the codomain $X_{n+1}$ of $\smap{n+1}$ as well
% as the codomain $X_{n-1}$ of $\dmap{n}{\_}$ must be non-empty.

Not all poset maps can be obtained as compositions of degeneracy and face maps
(in contrast to simplicial sets). Therefore we will shift our attention to
general poset maps $\sigma : \pint{m} \to \pint{n}$ in the following. 
Given an $n$-cell $p$, any such $\sigma$ gives rise to an $m$-cell
$X(\sigma)(p)$, which furthermore has the following faces:
$$\dmap{i}{e}(X(\sigma)(p)) = X(\restrict{\sigma}{\pintrestr{n}{i}{e}})(p)$$

TODO EXPLAIN WHY
Thereby we have a specified boundary.

In the next section we will see how this consideration is turned up side down:
Given a boundary, we want to come up with poset maps such as $\sigma$ to attain
the cell under question. For this we will introduce some notation. An
$n$-dimensional boundary $T$ is a list of $2n$ cells $\mlist{p_{(1,\izero)},
  p_{(1,\ione)} , ... , p_{(n,\izero)}, p_{(n, \ione)}}$. We will write
$T_{(i=e)} \mdef p_{(i,e)}$.

\paragraph{Kan cubical sets}


TODO Kan condition
This is not all! Kan condition

% In fact, we will be concerned in
% \autoref{sec:contortionsearch} with searching for intricate poset maps which
% give rise to cells with a specified boundary.


% A cubical set $X$ satisfies the Kan condition if for any open box
% $[t_1, s_1, \ldots , t_n , s_n] \ t$ we have a cell $s$ such that $[t_1,
% s_1, \ldots , t_n , s_n , t, s]$ is a valid boundary.\\[.5em]

% Call $\comp{[t_1, s_1, \ldots , t_n , s_n]}{t} \mdef s$

% When is a cubical set a \emph{type}, i.e., captures heterogenous identifications
% of elements in a type?

% Hope for syntax-free criterion.

% TODO Uniform Kan condition

% (still some issues: definitional equivalence of identity elimination has not
% been verified by model, only holds up to higher identification)
% A type is a cubical set satisfying the uniform Kan condition.

\begin{example}\label{exp:int}
  The cubical set $\cset{Int}$ is generated by the following data: $\cset{Int}_0
  = \{ \cset{zero} , \cset{one} \}$ and $\cset{seg} \in \cset{Int}_1$ with
  $\dmap{1}{0}(\cset{seg}) = \cset{zero}$ and $\dmap{1}{1}(\cset{seg}) =
  \cset{one}$. We have degenerate cells in higher dimensions for the $\smap{i}$
  to have a codomain, for instance, $\smap{1}(\cset{zero})$ is a degenerate
  1-cell which is constantly $\cset{zero}$, captured by the fact that the face
  maps $\dmap{1}{\_}$ send
  $\smap{1}(\cset{zero})$ back to $\cset{zero}$.

  Assuming that $\cset{Int}$ is Kan, we can obtain the reversal of $\cset{seg}$
  as follows:
  TODO Kan inversion example
\end{example}

\begin{example}\label{exp:loopspace}
  We define the cubical set $\cset{2Loop}$ with the following data: We have a
  single 0-cell, so $\cset{2Loop}_0 = \{
  \cset{a} \}$. There is one non-degenerate 2-cell $\cset{\alpha} \in
  \cset{2Loop}_2$, which has as its faces $\cset{a}$ as a degenerate 1-cell, i.e.,
  $\dmap{2}{0}(\cset{\alpha}) = \dmap{2}{1}(\cset{\alpha}) =
  \dmap{1}{0}(\cset{\alpha}) = \dmap{1}{1}(\cset{\alpha}) = \smap{1} (\cset{a})$.
\end{example}

% \begin{example}\label{exp:torus}
%   We define the cubical set $\cset{Torus}$ with the following data: We have a
%   single 0-cell, so $\cset{Torus}_0 = \{ \cset{a} \}$. There are two
%   non-degenerate 1-cells $\cset{p}$ and $\cset{q}$ and one non-degenerate 2-cell
%   $\dmap{2}{0}(\cset{\alpha}) = \dmap{2}{1}(\cset{\alpha}) = p$ and
%   $\dmap{1}{0}(\cset{\alpha}) = \dmap{1}{1}(\cset{\alpha}) = q$.
% \end{example}

\begin{example}\label{exp:triangle}
  We define the cubical set $\cset{Triangle}$ as generated by the following
  data: The 0-cells are $\cset{Triangle}_0 = \{ \cset{x} , \cset{y} , \cset{z}
  \}$. The non-degenerate 1-cells are $\{ \cset{p} ,
  \cset{q} , \cset{r} \} \subseteq \cset{Triangle}_1$ on which the face maps are defined by
    $\dmap{1}{0}(\cset{p}) = \cset{x}$ and $\dmap{1}{1}(\cset{p}) = \cset{y}$,
    $\dmap{1}{0}(\cset{q}) = \cset{y}$ and $\dmap{1}{1}(\cset{q}) = \cset{z}$,
    $\dmap{1}{0}(\cset{r}) = \cset{x}$ and $\dmap{1}{1}(\cset{r}) = \cset{z}$.
    We have one non-degenerate 2-cell $\cset{\phi} \in \cset{Triangle}_2$ with $\dmap{1}{0}(\cset{\phi}) =
    \cset{p}$, $\dmap{1}{1}(\cset{\phi}) = \cset{r}$, $\dmap{2}{0}(\cset{\phi})
    = \smap{1}(\cset{x})$ and $\dmap{2}{1}(\cset{\phi}) = \cset{q}$.
\end{example}

% \begin{example}{\label{exp:assoc}}
%   Associativity of three paths
% \end{example}





\subsection{Cubical sets algorithmically}

(Kan) cubical sets as introduced in the previous section are infinite objects:
As soon as there is a single cell in a cubical set $X$, we have faces or
degeneracies in all dimensions, and more generally all sorts of cells induced by
the poset maps. In order to mechanically search for cells in a cubical set, we
need a finitary representation of cubical sets. The examples above suggest that
we only require a bit of generating data from which the other cells and maps can
be inferred in a unique way. TODO BACKGROUND This is what HITs are for In
\cite[Sect. 6.4]{bezem14_model_type_theor_cubic_sets}, strict groupoid turned
into cubical set with general construction Higher inductive types are
constructed as generalized pushout construction (free $\omega$-groupoid
construction)

Furthermore, we will introduce a term language to talk about the objects
generated from this finitary representation, with normal forms. Also we consider
complexity of boundary computation 


\begin{definition}
  A \mname{context} $\Gamma$ is a list of declarations $\mlist{ p_1 : T_1 ,
    \ldots , p_k : T_k}$. The cubical set $X$ generated by a context $\Gamma$
  has cells $p_i$ with boundaries $\boundary{p_i} = T_i$ for $1 \leq i \leq k$
  and all necessary other cells. We call $\ctxtdim{\Gamma} \mdef k$ the length
  of the context.
\end{definition}

\todo{Define valid context and describe algorithm checking validity

  Checking that a context is well-formed can be done in $\mathcal{O}(? k)$ and is
  implemented in \autoref{alg:wellformedboundary} and
  \autoref{alg:wellformedcontext}.
}



In the following we will assume to have given a valid context $\Gamma$
generating a cubical set $X$.

There are different ways of describing a cell in a cubical set. For our
considerations we will want a unique representation for every cell to allow for
literal comparisons between cells. For this we introduce the following notion:

\begin{definition}
  Given a poset map $\sigma : \pint{m} \to \pint{n}$ and an $n$-cell $p$, we
  will write $\cont{p}{\sigma} \mdef X(\sigma)(p)$ and call $\cont{p}{\sigma}$ an
  $m$-contortion of $p$.
\end{definition}

For example, the degenerate 1-cell $\smap{1}(\cset{zero})$ from
\autoref{exp:int} is represented in our setting by the 1-contortion
$\cont{\cset{zero}}{\oneconst}$, where we directly depict the graph of the poset
map. The identity map gives an $n$-contortion for every $n$-cell which is equal
to the original cell, for instance $\cont{\cset{seg}}{\oneid}$ is just the cell
$\cset{seg}$. Even though we describe every cell as a contortion in the
following, we might omit the poset map if it is the identity.

However, there are still multiple ways to describe the same cell: we could
have also used $\cont{\cset{seg}}{\substtwo{0}{0}}$ to describe the degenerate
1-cell which is constantly $\cset{zero}$. Since $\cont{\cset{zero}}{\oneconst}$
is more basic, we will regard it as the normal form of this cell. In general:

\begin{definition}
  A contortion $\cont{p}{\sigma}$ is in normal form if there is no face $q$ of
  $p$ and map $\sigma'$ such that $\cont{p}{\sigma} = \cont{q}{\sigma'}$.
\end{definition}


We have an algorithm reducing a contortion to normal form as follows:

\begin{algorithm}[H]
  \caption{Normalizing a contortion}
  \label{alg:normalize}
  \begin{algorithmic}
    \Require $\cont{p}{\sigma}$ where $p : T \in \Gamma$ is an $n$-cell and
    $\sigma : \pint{m} \to \pint{n}$ \Ensure Normal form of $\cont{p}{\sigma}$
    
    \Procedure{Normalize}{$\cont{p}{\sigma}$}
    \If{$\image{\sigma} \subseteq \pintrestr{n}{i}{e}$ for some $1 \leq i \leq n$ and $e
      \in \{\izero, \ione\}$}
    \State $\cont{q}{\sigma'} \gets T_{i=e}$
    \State \Return{ $\Call{Normalize}{\cont{q}{\sigma' \circ \smap{i} \circ \sigma }}$ }
    \Else{}
    \State \Return{$\cont{p}{\sigma}$}
    \EndIf
    \EndProcedure

  \end{algorithmic}
\end{algorithm}

TODO EXPLAIN
The poset map $\sigma' \circ \smap{i} \circ \sigma $ comes about as follows: We
know that  $\sigma(x)_i = e$ for all $x \in \pint{m}$. Therefore we consider
where the $(i,e)$-th face of $p$ is sending all elements after forgetting about
this index.

Checking whether a subset of a poset is of the form $\pintrestr{n}{i}{e}$ can be
done in $\mathcal{O}(2^n)$ since minimal distance TODO EXPLAIN?

Since the dimension of the contortion under question is decreased by 1 in each
normalization step, we need to run the algorithm at most $n$ times and therefore
have an asymptotic run time of $\mathcal{O}(2^n)$.

\autoref{alg:normalize}



\todo{where to describe \autoref{alg:boundary} }



% TERM DEFINITION 
% Given a context $\Gamma$, the set of $m$-dimensional terms over $\Gamma$, denoted $Tm_m(\Gamma)$, consists of
% \begin{itemize}
% \item Contortions $\cont{p}{\sigma}$ with $p \in \Gamma$ and $\sigma : \pint{m} \to
%   \pint{\dim{p}}$
%   % \item Kan compositions, i.e., $\comp{[(t_1 , s_1) , (t_m, s_m)]}{t}$ for all
% \item Kan compositions, i.e., $\comp{[t_1 , s_1, ... , t_m, s_m]}{t}$ for all
%   fitting $m$-contortions $t_1 , s_1, ... , t_m, s_m, t$.
% \end{itemize}


\begin{examplecontd}{exp:int}
  The following context generates $\cset{Int}$:
  
  $\mlist{ \cset{zero} : \mlist{} , \cset{one} : \mlist{} , \cset{seg} : \mlist{
      \cont{\cset{zero}}{}, \cont{\cset{one}}{} }}$
\end{examplecontd}

\begin{examplecontd}{exp:loopspace}
  The following context generates $\cset{2Loop}$:

  $\mlist{ \cset{a} : \mlist{} , \cset{\alpha} : \mlist{ \cont{\cset{a}}{\oneconst} , \cont{\cset{a}}{\oneconst} ,
  \cont{\cset{a}}{\oneconst}, \cont{\cset{a}}{\oneconst} }}$
  
\end{examplecontd}

\begin{examplecontd}{exp:triangle}
  The following context generates $\cset{Triangle}$:
  
  % $\mlist{ \cset{x} : \mlist{} , \cset{y} : \mlist{} , \cset{z} : \mlist{} ,
  %     \cset{p} : \mlist{ \cont{\cset{x}}{}, \cont{\cset{y}}{} } ,
  %   , \cset{q} : \mlist{ \cont{\cset{y}}{}, \cont{\cset{z}}{} }
  %   , \cset{r} : \mlist{ \cont{\cset{x}}{}, \cont{\cset{z}}{} }
  %   , \cset{\phi} : \mlist{ \cont{\cset{p}}{\oneid} , \cont{\cset{r}}{\oneid} ,
  %     \cont{\cset{x}}{\oneconst}, \cont{\cset{q}}{\oneid} }
  % }$

  $\mlist{ \cset{x} : \mlist{} , \cset{y} : \mlist{} , \cset{z} : \mlist{} ,
    \cset{p} : \mlist{ \cset{x}, \cset{y}  } ,
    , \cset{q} : \mlist{ \cset{y}, \cset{z} }
    , \cset{r} : \mlist{ \cset{x}, \cset{z} }
    , \cset{\phi} : \mlist{ \cset{p} , \cset{r} ,
      \cont{\cset{x}}{\oneconst}, \cset{q} }
  }$
  
  The contortion $(\phi,\substfour{00}{00}{01}{01})$ has normal form $(p,\substfour{0}{0}{1}{1})$.
\end{examplecontd}

\subsection{Searching for cells}

The search problem is the following: given a cubical set and a certain boundary,
is there a cell with that boundary. Using the finitary representation introduced
above, we can phrase it as follows:
  
\begin{definition}
  Given a context $\Gamma$ and a (valid) boundary $T$, the problem
  \problem{CubicalCell}($X$,$T$) is to find a term $t \in Tm_{\dim{T}}(\Gamma)$ such that
  $\boundary{t} = T$.
\end{definition}

(More generally, we can encode all word problems from universal algebra in our setting.)

\begin{theorem}
  \problem{CubicalCell} is undecidable.
  \begin{proof}
    
    % Given a set of generators $\{a, ...\}$ of a group $G$, define $\cset{Group}$
    % with a single 0-cell $\cset{Group}_0 = \{ \cset{\star} \}$, 
    % 1-cells $\{\cset{a} , \cset{a^{-1}} ... \} \subseteq \cset{Group}_1$ and a
    % 2-cell $\cset{idem_a}$ for any generator $a$ with 
    % $\dmap{2}{0}(\cset{idem_a}) = \cset{a}$, $\dmap{1}{1}(\cset{idem_a}) =
    % \cset{a^{-1}}$ and $\dmap{2}{1}(\cset{idem_a}) = \dmap{1}{0}(\cset{idem_a}) = \smap{1} (\cset{\star})$.

    Given a set of generators $\{a, ...\}$ of a group $G$, define $\cset{Group}$
    with a single 0-cell $\cset{Group}_0 = \{ \cset{\star} \}$, 
    1-cells $\{\cset{a} , ... \} \subseteq \cset{Group}_1$ 
    
    (there are more 2-cells that need to be added as explained in \cite[Sect. 6.3]{bezem14_model_type_theor_cubic_sets})
    % Note that this is not a set. Doesn't matter for undecidability proof?

\end{proof}
\end{theorem}

If it can be solved by a contortion, it is clearly decidable since there are
only finitely many substitutions for each cell in the context, and the search
space is thereby finite. However, the search space grows super-exponentially:
Suppose we want to check if there is an $m$-dimensional contortion of a 1-cell
satisfying a goal, i.e., we are looking for a monotone map $\pint{m} \to
\pint{}$,. For $m = 6$, there are $7828354$ substitutions that we would have to
check and for 9 the exact number of possible substitutions has not even been
computed yet \url{http://oeis.org/A00372}. Therefore we need something more
clever to explore the search space.



\section{Decidable \problem{CubicalCell} / Searching for contortions}

A certainly decidable subproblem of \problem{CubicalCell} is to decide whether
there exists a cell simply by means of contorting the cells given in the
context: for a given goal boundary $T$ and a generating cell $p$, there are
finitely many poset maps $\pint{\dim{T}} \to \pint{\dim{p}}$, so we can
na\"ively check whether one of those maps gives rise to a fitting contortion. 

We can try this for all elements in the context, thereby have run time
$\mathcal{O}( \ctxtdim{\Gamma} )$

\subsection{General complexity considerations}


Revert characterization 
$$\dmap{i}{e}(X(\sigma)(p)) = X(\restrict{\sigma}{\pintrestr{n}{i}{e}})(p)$$


\subsection{Potential substitutions}

For an effective algorithm, we turn to following notion.

We introduce potential substitution, which keep track of all the potential
values of $\pint{n}$ a poset map $\sigma$ might assign to an element of $x \in \pint{m}$.

They allow to represent all maps $\pint{m} \to \pint{n}$ at once.
Solves memory problem. We need to explore the search space cleverly.

\begin{definition}
  A \mname{potential poset map} (\mname{ppm}) is a map $\Sigma : \pint{m} \to \pow{\pint{n}}$
  such that for all $x \leq y$ : for all $u \in \Sigma(x)$ exists $v \in
  \Sigma(y)$ such that $u \leq v$.
\end{definition}

The condition on poset maps means that 


For effectively computing with poset maps we will consider their graphs. We also
list the vertices such that if $x \leq y$ then $x$ is earlier than $y$ in the
list. This allows us to traverse through the graph and know that if an element
$x$ is at a certain position in the list, all elements below $x$ will be seen at
a certain point.

A non-empty potential map is one which has a non-empty set of possible values
for any element, which can be checked in $2^m$ steps. Given a poset $xs
\subseteq \pint{m}$, we can restrict a ppm $\Sigma : \pint{m} \to \pow{\pint{n}}$ to a
new ppm $\restrict{\Sigma}{xs} : xs \to \pow{\pint{n}}$ just like any function.

\autoref{alg:updateppm}
Therefore we have a runtime of $\mathcal{O}(2^m 2^n)$

From a potential poset map we can generate all poset maps. This has a worst-case
runtime of Dedekind for the full one.


\autoref{alg:getsubsts}
$\mathcal{O}(D_m^n)$


\subsection{Simple solver}

We want to try and see 

\begin{algorithm}[H]
  \caption{Simple solver}\label{alg:simple}
  \begin{algorithmic}
    \Require $\Gamma$ context, $p \in \Gamma$, $T$ goal boundary
    \Ensure $\sigma$ s.t. $\boundary{\cont{p}{\sigma}} = T$ if there is such a
    $\sigma$, \texttt{Unsolvable} otherwise

    \Procedure{SimpleSolve}{$\Gamma, T$}
    \State $m = \dim{T} , n = \dim{S} $
    \State $\Sigma \gets \{ x \mapsto \pint{n} \mid x \in \pint{m} \}$
    \For{$xs \gets (m-1)$-face} \Comment{$2 \cdot m$ such faces}
    \State $\sigma s \gets $ \Call{UnfoldPPM}{$\restrict{\Sigma}{xs}$} \Comment{$D_{m-1}^n$}
    \For{$x \gets xs$}
      \State \Call{UpdatePPM}{$\Sigma, x, \{ v \mid v \in \Sigma(x) : \exists
        \sigma \in \sigma s : \sigma(x) = v, $
      \Statex \hfill \Call{Normalize}{$p,img(\sigma)$} $ = {T}@ xs \}$}
    \EndFor
    \EndFor
    \If{$\exists \sigma \in \Call{UnfoldPPM}{\Sigma} : \boundary{\cont{p}{\sigma}} = T$}
      \State \Return{$\sigma $}
    \Else
    \State \Return{\texttt{Unsolvable}}
    \EndIf
    \EndProcedure

    % let psubst = foldr
    % (\xs sigma ->
    % let a = evalBoundary cube goal xs in
    % let gadgets = map (map snd . Map.toList) (getSubsts (sigma `restrictKeys` Set.fromList xs)) in
    % let gadgets' = filter (\g -> evalFace cube f g == a) gadgets in
    % let vus = map (\i -> nub (map (!!i) gadgets')) [0 .. length xs - 1] in
    % foldl (\s (x , vu) -> updatePSubst s x vu) sigma (zip xs vus)
    % )
    % (createPSubst (dim goal) (dim ty))
    % (getFaces (dim goal) (dim goal - 1)) in
  \end{algorithmic}
\end{algorithm}

Instead of $D_m^n$ runtime we have $2 m D_{m-1}^n$ runtime, which is a significant
improvement.

More than that, the potential substitutions allow us to narrow down the search
space in other ways.

Try to reduce the search space going from the bottom.

COMPLETENESS???
It's complete. But not correct -- overapproximating results





\section{Composition solver}

We use Kan composition



\section{Negation}

How to encode search for reversals


\section{Relation to different notions of cubical sets}

Since the generating data is the non-degenerate cells and the face maps on them,
we can adopt the following notation to describe cubical sets, which go under
higher inductive types in Cubical Agda:

TODO also boundary description with poset maps. These correspond to specifying
the dmaps: One index is constant, rest is given by given poset map


\subsubsection{Relation to face formulas}

The morphisms in $\dedekind$ have a succinct description as tuples of elements of a free
distributive lattices, which we will call telescopes.
Cubical Agda uses telescopes to describe this. However, these are hard to
construct and have no geometric intuition. This is why we had used poset maps.
We can go back and forth between both representations easily.


Given $s : \pint{m} \to \pint{n}$, we compute an $n$-tuple of elements of the
free distributive lattice over $m$ elements as follows:

The $i$-th entry is $\{ x \in \pint{m} \mid s(x)_i = 1 \}$. An element $x$ can be
seen as a clause if we regard it as indicator of which elements of the lattice
are used, e.g., $(1,0,1)$ represents the clause $x \meet z$ if the three
elements of the lattice are $x$, $y$ and $z$.

\begin{algorithm}[H]
  \caption{TODO}\label{alg:subst2tele}
  \begin{algorithmic}
    \Require $\sigma : \pint{m} \to \pint{n}$
    \Ensure $\phi$ $n$-tuple of elements of free distributive lattice over $m$ variables

    \Procedure{Subst2Tele}{$\sigma$}
    \For{$i \gets 1$ to $n$} 
      \State $\phi_i \gets \{ x \in \pint{m} \mid \sigma(x)_i = 1 \}$
      \Comment{$\mathcal{O}(2^m)$ many elements in $\pint{m}$}
    \EndFor
    \State \Return{$(\phi_1, ... , \phi_n)$}
    \EndProcedure
  \end{algorithmic}
\end{algorithm}

TODO mention set representation of DNF. Also mention normalization necessary --
what's its runtime?

From an $n$-tuple of formulas over $\phi$ we can read off a poset map $s :
\pint{m} \to \pint{n}$ as follows: Given an element $x \in \pint{m}$, $s(x) = (e_1
, ... , e_n) $ where $e_i = \phi_i @ x$ TODO EVALUATION

\begin{algorithm}[H]
  \caption{TODO}\label{alg:teletosubst}
  \begin{algorithmic}
    \Require TODO
    \Ensure TODO

    \Procedure{Tele2Subst}{$p$}
    \For{$x \gets \pint{m}$} 
      \For{$i \gets 1$ to $n$}
        \State $\sigma(x)_i \gets \phi_i @ x$
      \EndFor
    \EndFor
    \State \Return{$\sigma$}
    \EndProcedure
  \end{algorithmic}
\end{algorithm}


Therefore going back and forth between formulas and poset map takes
$\mathcal{O}(2^mn)$ many steps. This is linear in the number of elements of the
data structures we are considering, so pretty quick and no obstruction.

\begin{example}
  % EXAMPLE
  % [(000,00),(001,00),(010,00),(011,01),(100,10),(101,11),(110,10),(111,11)]
  % 1 ((2 /\ 3) \/ (1 /\ 3))
\end{example}

TODO PROOF THAT THESE ARE MUTUALLY INVERSE?


\subsection{Nominal perspective}

Have set $I = \{i , j, k , ...\}$ of names. Then give complete description for
Cubical Agda.


\appendix

\section{Algorithms}

We will give pseudo-code for algorithms. We assume we have a context $\Gamma$
generating a cubical set and a generating cell $p \in \Gamma$ with $\dim{p} = n$ in the
following.

TODO definition retrieval?

\begin{algorithm}[H]
  \caption{Computing the face of a contortion}
  \label{alg:boundary}
  \begin{algorithmic}
    \Require Contortion $\cont{p}{\sigma}$ with $\sigma : \pint{m} \to
    \pint{n}$, $(i,e)$  $1 \leq i \leq n$, $e \in \{\izero, \ione \}$
    \Ensure $\dmap{i}{e}(\cont{p}{\sigma})$
    
    \Procedure{Face}{$\cont{p}{\sigma}, (i,e)$}
    \State Do Something
    % let facepos = createPoset (domdim sigma - 1) in
    % Boundary $ map (\i -> (
    % evalFace ctxt p (map (\x -> sigma ! insInd i e0 x) facepos),
    % evalFace ctxt p (map (\x -> sigma ! insInd i e1 x) facepos)))
    % (reverse [0 .. domdim sigma - 1])
    \EndProcedure
  \end{algorithmic}
\end{algorithm}


\begin{algorithm}[H]
  \caption{Computing the boundary of a contortion}
  \label{alg:boundary}
  \begin{algorithmic}
    \Require Contortion $\cont{p}{\sigma}$ where $p$ is an $n$-cell of $X$ and $\sigma : \pint{m} \to \pint{n}$.
    \Ensure $\boundary{\cont{p}{\sigma}}$
    
    \Procedure{Boundary}{$\cont{p}{\sigma}$}
    \State Do Something
    % let facepos = createPoset (domdim sigma - 1) in
    % Boundary $ map (\i -> (
    % evalFace ctxt p (map (\x -> sigma ! insInd i e0 x) facepos),
    % evalFace ctxt p (map (\x -> sigma ! insInd i e1 x) facepos)))
    % (reverse [0 .. domdim sigma - 1])
    \EndProcedure
  \end{algorithmic}
\end{algorithm}



% \begin{algorithm}[H]
%   \caption{Normalize a substituted term to normal form}
%   \label{alg:normalize}
%   \begin{algorithmic}
%     \Require $(p, vs)$ where $p$ is an $n$-cell of $X$/$Gamma$ and $vs$ multiset
%     of $\pint{n}$
%     \Ensure $(f, \sigma)$ normalized term if face of original term
    
%     \Procedure{Normalize}{$p, vs$}
%     \If{$vs$ is sub face}
%     \State Do Something
%     \Else{}
%     \State $\sigma : \pint{n} \to \pint{n}, \sigma(x) \to vs !! i$
%     \State \Return{$(p, \sigma)$}
%     \EndIf
%     \EndProcedure

%     % let ty = lookupDef cube f in
%     % case dim ty of
%     % 0 -> Term f (constSubst (log2 (length vs)))
%     % n -> if any (\u -> head vs `vdiff` u > n-1) (tail vs)
%     % then Term f (reconstrPMap vs)
%     % else evalBoundary cube ty vs
%   \end{algorithmic}
% \end{algorithm}



\begin{algorithm}[H]
  \caption{Well-formed boundary}\label{alg:wellformedboundary}
  \begin{algorithmic}
    \Require $\Gamma$ context, $T$ term
    \Ensure \texttt{OK} if $T$ well formed, \texttt{Error} otherwise

    \Procedure{WellFormed}{$\Gamma, T$}
    \State Do Something
    \EndProcedure
  \end{algorithmic}
\end{algorithm}

\begin{algorithm}[H]
  \caption{Well-formed context}\label{alg:wellformedcontext}
  \begin{algorithmic}
    \Require $\Gamma$ context
    \Ensure \texttt{OK} if $\Gamma$ well formed, \texttt{Error} otherwise

    \Procedure{WellFormed}{$\Gamma$}
    \State Gradually build up? Or is it ok to have mutually defined cells?
    \EndProcedure
  \end{algorithmic}
\end{algorithm}




\begin{algorithm}[H]
  \caption{Faces}\label{alg:faces}
  \begin{algorithmic}
    \Require $m$, $n$
    \Ensure $\{ xs \}$ where $xs$ are the $n$-faces of $\pint{m}$

    \Procedure{Faces}{$m,n$}
    \If{$n = 0$}
    \State $\{ \{ x \} \mid x \in \pint{m} \}$
    \ElsIf{$m = n$}
    \State $\{ \pint{m} \}$
    \Else
    \State $\{ \{ 0x \mid x \in xs \} , \{ 1x \mid x \in xs\} \mid xs \in $ \Call{Faces}{$m-1,n$} $ \}$
    \Statex $\cup TODO$
    \EndIf
    \EndProcedure
    
    % getFaces :: Int -> Int -> [[Vert]]
    % getFaces m 0 = map (: []) (createPoset m)
    % getFaces m n | m == n = [ createPoset m ]
    % getFaces m n =
    % map (map (e0 `insv`)) (getFaces (m-1) n)
    % ++ map (map (e1 `insv`)) (getFaces (m-1) n)
    % ++ map (\l -> map (e0 `insv`) l ++ map (e1 `insv`) l) (getFaces (m-1) (n-1))
  \end{algorithmic}
\end{algorithm}



\subsection{Potential substitutions}


\begin{algorithm}[H]
  \caption{Update potential poset map}\label{alg:updateppm}
  \begin{algorithmic}
    \Require $\Sigma : \pint{m} \to \pow{\pint{n}}$ ppm, $x \in
    \pint{m}$, $vs \in \pow{\pint{n}} - \emptyset$
    \Ensure Updated ppm $\Sigma' : \pint{m} \to \pint{n}$ with $\Sigma(x) = vs$.

    \Procedure{UpdatePPM}{$\Sigma, (x,vs)$}
    \For{$y \gets \pint{m}$}
    \If{$x = y$}
    \State $\Sigma'(x) \gets vs$
    \ElsIf{$y \leq x$}
    \State $\Sigma'(y) \gets \{ u \mid u \in \Sigma(y) \text{ such that } \exists
    v \in vs : u \leq v \} $
    \ElsIf{$x \leq y$}
    \State $\Sigma'(y) \gets \{ u \mid u \in \Sigma(y) \text{ such that } \exists
    v \in vs : v \leq u \} $
    \EndIf
    \EndFor
    \State \Return{$\Sigma'$}
    \EndProcedure
  \end{algorithmic}
\end{algorithm}



\begin{algorithm}[H]
  \caption{Potential substitution to substitutions}\label{alg:getsubsts}
  \begin{algorithmic}
    \Require $\Sigma : \pint{m} \to \pow{\pint{n}}$ potential poset map
    \Ensure $\{ \sigma : \pint{m} \to \pint{n} \}$ poset maps
    
    \Procedure{UnfoldPPM}{$\Sigma$}
    \State $\sigma : x \mapsto \Sigma(x) $
    TODO
    \EndProcedure
  \end{algorithmic}
\end{algorithm}



% For an efficient algorithm we instead consider an algorithm which extracts the
% first substitution from a potential substitution.

% \begin{algorithm}[H]
%   \caption{Potential substitution to substitution}\label{alg:simple}
%   \begin{algorithmic}
%     \Require $\Sigma : \pint{m} \to \pow{\pint{n}}$ non-empty ppm
%     \Ensure $\sigma : \pint{m} \to \pint{n}$ poset map

%     \Procedure{FstPPM}{$\Sigma$}
%     \For{$x \gets \pint{m}$}
%     \State $v \gets \mathsf{first}(\Sigma(x))$
%     \State $\sigma \gets \Call{FstPPM}{\restrict{\Sigma}{-x}}$
%     \State $\sigma(x) \mapsto v$
%     \EndFor
%     \EndProcedure
%   \end{algorithmic}
% \end{algorithm}

% This has amortised runtime of $\mathcal{O}(2^m 2^n)$.

\end{document}